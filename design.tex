\documentclass{article}
\usepackage{graphicx}

\begin{document}

\title{CS425 MP1 \\ \large Distributed chat client using causal ordering.}
\author{Mike Crawford and Sean Nicholay}

\maketitle

\section{Introduction}
The goal of this MP was to extend the skeletal chat client to include causal ordering, reliable multicast and failure detection.  

\subsection{Causal Ordering}

\subsection{Reliable Multicast}
We created a reliable multicast by wrapping the underlying unicast methods with a reliability layer.  The basic operation of the reliabilty layer is for each unicast message sent out we have the recipient reply with an acknowledgement, and if the sender does not get the acknowledgement it will resend the message.  

Upon issuing the unicast send request the message will be initally sent and copied into a queue.  A second thread monitors the queue and will watch to see when a message is ready for retransmission.  If it hasn't been removed by that point it will be retransmitted.  If the number of retransmissions hits a set point and it has not been acknowledged by a member, we consider that process as failed and remove it from the list.

Upon receiving a message the process will send an acknowldgement packet.  This will then get received by the original sending process.  That process will then remove the queued message from its send queue.  At that point it will be considered sent and no more retries will be attempted.

Duplicate messages are also filtered from message delivery.  A message history is kept that contains the message sequence number and the message source.  Upon receiving a dupliacte message, the system will acknowledge the message and not deliver the duplicate to the causal re-order function.

\subsection{Failure Detection}
For our failure detection we implemented a multicast ping where each process will multicast a ping to all processes.  Upon a process getting a ping it will send an acknowledgement, and this ackowledgement will cause the sender to remove the message from the send queue and the host will be considered up.  If the message never gets recieved within a timeout the process will try to ping again.  After a specified amount of timeouts the process will give up and fail the process.  To reduce chatter on the network if any other message is recieved from the host we consider any active pings to be served.  This will catch most cases, but in some edge cases it may require another round of pings to detect a failure.


\end{document}
